\documentstyle[11pt]{article}
\renewcommand{\baselinestretch}{1.3}
\newcommand{\itemspace}{\hspace{8pt}}   
\newcommand{\hfhf}{\left(\frac{1}{2},\frac{1}{2}\right)}
\newcommand{\overp}{\overline{P}}
\def\thefigure{\arabic{section}.\arabic{subsection}.\arabic{figure}[p]}

\def\mod{{\rm mod}}
%%%%%%%%%%%%%%%%%%%%%%%%%%%%%%%%%%%%%%%%%%%%%%
%%%
%%%      Welcome to the macros of Ilan Vardi!
%%%
%%%
%%%%%%%%%%%%%%%%%%%%%%%%%%%%%%%%%%%%%%%%%%%%%%
%%%%%%
%%%%%%      1 in margins produced by this 
%%%%%%
%%%%%%%%%%%%%%%%%%%%%%%%%%%%%%%%%%%%%%%%%%%%%%%%
\setlength{\textwidth}{6.5in}                                
\setlength{\textheight}{8.9in}                               
\setlength{\topmargin}{0pt}                          
\setlength{\oddsidemargin}{0pt}                          
\setlength{\evensidemargin}{0pt}                         
\setlength{\headheight}{0pt}                             
\setlength{\headsep}{0pt}    
%%%%%%%%%%%%%%%%%%%%%%%%%%%%%%%%%%%%%%%%%%%%%%%%%%%%%%%%%%%
%%%%%%%%%%%%%%%%%%%%%%%%%%%%%%%%%%%%%%%%%%%%%%%%%%%%%%%%%%%%%%%%
%%%
%%%      TeX's eqalign macros....
%%%
\catcode`@=11
\newskip\@centering \@centering=0pt plus 1000pt minus 1000pt
\def\openup{\afterassignment\@penup\dimen@=}
\def\@penup{\advance\lineskip\dimen@
  \advance\baselineskip\dimen@
  \advance\lineskiplimit\dimen@}
\def\eqalign#1{\null\,\vcenter{\openup\jot\m@th
  \ialign{\strut\hfil$\displaystyle{##}$&$\displaystyle{{}##}$\hfil
      \crcr#1\crcr}}\,}
\newif\ifdt@p
\def\displ@y{\global\dt@ptrue\openup\jot\m@th
  \everycr{\noalign{\ifdt@p \global\dt@pfalse
      \vskip-\lineskiplimit \vskip\normallineskiplimit
      \else \penalty\interdisplaylinepenalty \fi}}}
\def\@lign{\tabskip\z@skip\everycr{}} % restore inside \displ@y
\def\displaylines#1{\displ@y
  \halign{\hbox to\displaywidth{$\@lign\hfil\displaystyle##\hfil$}\crcr
    #1\crcr}}
\def\eqalignno#1{\displ@y \tabskip\@centering
  \halign to\displaywidth{\hfil$\@lign\displaystyle{##}$\tabskip\z@skip
    &$\@lign\displaystyle{{}##}$\hfil\tabskip\@centering
    &\llap{$\@lign##$}\tabskip\z@skip\crcr
    #1\crcr}}
\def\leqalignno#1{\displ@y \tabskip\@centering
  \halign to\displaywidth{\hfil$\@lign\displaystyle{##}$\tabskip\z@skip
    &$\@lign\displaystyle{{}##}$\hfil\tabskip\@centering
    &\kern-\displaywidth\rlap{$\@lign##$}\tabskip\displaywidth\crcr
   #1\crcr}}
\catcode`@=12
%%%%%%%%%%%%%%%%%%%%%%%%%%%%%%%%%%%%%%%%%%%%%%%%%%%%%%%%%%
%%%%%   boxed text
%%%%%%%%%%%%%%%%%%%%%%%%%%%%%%%%%%%%%%%%%%%%%%%%%%%%%%%%%%

\newlength{\boxedparwidth}
 \setlength{\boxedparwidth}{.92 \textwidth}

\newenvironment{boxedtext}
{\begin{center}
 \begin{tabular}{|@{\hspace{.15 in}}c@{\hspace{.15 in}}|}
  \hline \\ \begin{minipage}[t]{\boxedparwidth}
    \setlength{\parindent}{.25 in}}
    {\end{minipage} \\ \\ \hline \end{tabular} \end{center}}
%%%%%%%%%%%%%%%%%%%%%%%%%%%%%%%%%%%%%%%%%%%%%%%%%%%%%%%%%%%%%%%%
\def\calc{{\cal C}}

\begin{document}
\begin{center}
{\LARGE Code and pseudo code}


\medskip
{\Large Ilan Vardi}

\medskip
Appeared in The Mathematica Journal {\bf 6} Issue 2 (1996), 66--71.

\end{center}






\bigskip
It is quite common that mathematical results rely on explicit
descriptions of algorithms. These are often written in what is called
``pseudo code'' usually based on a computer language like Pascal or C
\cite{Gonnet} \cite{Sedgewick} \cite{SedgewickC}.  
The question addressed here is
how to check that pseudo code is correct, in other words, how do you
get a ``pseudo compiler'' to check your pseudo code? I propose that a
way to do this is to use a high level language such as {\sl
Mathematica\/} to write a real implementation. The high level and
built--in mathematical capabilities of such a language allow you to
write pseudo code that is essentially a transcription of the computer
code, and so hopefully free of ``pseudo bugs.''

\medskip\noindent
{\bf Acknowledgments.} I would like to thank Henry Cejtin and Igor Rivin
for helpful comments. 

\subsection*{1. A simple example}

Consider the problem of adding and multiplying positive integers by
using their decimal expansions, i.e., describe algorithms that given
two strings of digits representing two integers, return the strings of
digits corresponding to the sum and product of the numbers.

The input will be sequences of digits $[d_n,\ldots,d_0]$.
Concatenation $x\star y$ of sequences will be used along with the
unusual convention that $[0]\star x$ returns $x$.
This condition corresponds to the usual convention that omits
leading zeroes in decimal expansions.

\medskip\noindent
\begin{boxedtext}
\noindent
{\bf Addition algorithm:} Given as input two sequences of
digits $[a_m,\ldots,a_0]$ and $[b_n,\ldots,b_0]$ return a sequence
$a\oplus b$:

\medskip\noindent
{\bf intialize} $k\Leftarrow 0$

\smallskip\noindent
{\bf for} $i=0,1,\ldots, \max(m,n)$ {\bf do}

\smallskip
$k \Leftarrow a_i + b_i + \lfloor k/10\rfloor$. If $i > m$, replace
$a_i$ with $0$ and if $i> n$, replace $b_i$ with $0$

\smallskip
$c_i \Leftarrow k\; {\rm mod}\; 10$

\smallskip\noindent
{\bf return} $[\lfloor k/10 \rfloor]\star [c_{\max(m,n)},\ldots,c_0]$
\end{boxedtext}

\medskip\noindent 
\begin{boxedtext}
\noindent
{\bf Multiplication algorithm:} Given as input two sequences of
digits $[a_m,\ldots,a_0]$ and $[b_n,\ldots,b_0]$ return a sequence 
$a\otimes b$:

\smallskip\noindent
{\bf initialize} $c \Leftarrow [\;]$ 

\noindent
{\bf for} $i=0,1,\ldots,m$ and $j=0,1,\ldots,n$ 
{\bf do}
$
c \Leftarrow c \oplus 
([\lfloor a_i b_j/10\rfloor]\star [a_ib_j\;{\rm mod}\; 10,
\,\overbrace{0,0,\ldots,0}^{i+j}\,]) 
$

\smallskip\noindent
{\bf return} $c$
\end{boxedtext}


\medskip\noindent
The {\sl Mathematica\/} implementation starts by defining
the $\star$ operator

{\sl 
\begin{verbatim}
x_~join~y_:= If[x == {0}, y, x~Join~y]
\end{verbatim}
}

\noindent
The {\sl Mathematica\/} statement

\smallskip
{\tt If[{\sl condition}, {\sl statement1}, {\sl statement2}]}

\smallskip\noindent
means the same as 

\smallskip
{\bf if} {\sl condition} {\bf then} 
{\sl statement1} {\bf else} {\sl statement2}

\smallskip\noindent
The infix notation \verb+x~join~y+, meaning the 
same as \verb+join[x, y]+, has been used in order to make the 
program look more like the mathematical description of the algorithm.

The next thing is to write functions that take a list 
$[d_k,\ldots,d_0]$ and return the last element $d_0$, and the list
without its last element, i.e., $[d_k,\ldots,d_1]$. 

{\small
\begin{verbatim}
last[x_]:= {Last[x]}

drop[x_]:= Drop[x, -1]
\end{verbatim}
}

\noindent
The addition and multiplication programs are then

\begin{boxedtext}
\noindent
{\small
\begin{verbatim}
a_~plus~b_:= 
Block[{c = {}, i, k = 0},
       Do[k = If[i > Length[a], 0, a[[-i]]] + If[i > Length[b], 0, b[[-i]]] +
              Floor[k/10];
          PrependTo[c, Mod[k, 10]],
          {i, 1, Max[Length[a], Length[b]]}];
       Return[{Floor[k/10]}~join~c]
      ]
\end{verbatim}
}
\end{boxedtext}

\medskip
\begin{boxedtext}
\noindent
{\small 
\begin{verbatim}
a_~times~b_:= 
Block[{c = {}, i, j, k},
      Do[k = a[[-i]] b[[-j]]; 
         c = c~plus~({Floor[k/10]}~join~Prepend[Table[0,{i+j-2}], Mod[k,10]]),
         {i, 1, Length[a]}, {j, 1, Length[b]}];
      Return[c]
     ] 
\end{verbatim}
}
\end{boxedtext} 

\medskip\noindent
Note that {\sl Mathematica\/} has a somewhat idiosyncratic syntax, for
example, the fact that arrays can be accessed from the right using
negative indices, or expressions like 

\begin{verbatim}
Quotient[#, 10]~join~Mod[#,10]& [last[a] last[b]], 
\end{verbatim}

\noindent
used below. On the other hand, the control structure of the program is
identical to the high one in the high level description and once you
get used to the peculiarities of the language, the high level
description can be directly transcribed from the program.

An even better example is to use a recursive
algorithm

\medskip\noindent
\begin{boxedtext}
\noindent
{\bf Addition algorithm:} Given as input two sequences of
digits $a = [a_m,\ldots,a_0]$ and $b=[b_n,\ldots,b_0]$,
return a sequence $a\oplus b$:

\begin{description}

\item[{\bf if}]
$a=[\;]$ or $b=[\;]$ {\bf then} $a\star b$

\item[{\bf else}]
$
(([a_m,\ldots,a_1]\oplus [\lfloor (a_0+b_0)/10\rfloor])\oplus 
 [b_n, \ldots,b_1]) \star [(a_0+b_0)\;{\rm mod}\; 10]
$
\end{description}
\end{boxedtext}

\medskip\noindent
\begin{boxedtext}
\noindent
{\bf Multiplication algorithm:} Given as input two sequences of
digits $a = [a_m,\ldots,a_0]$ and $a = [b_n,\ldots,b_0]$ return 
a sequence $a\otimes b$: 

\begin{description}

\item[{\bf if}] $n = 0$ {\bf then}

\begin{description}

\item[{\bf if}] $m=0$ {\bf then} 
$[\lfloor a_0 b_0/10\rfloor]\star [a_0 b_0\;{\rm mod}\; 10]$

\item[{\bf else}] $b\otimes a$

\end{description} 

\item[{\bf else}] 
$((a\otimes [b_n,\ldots,b_1])\star [0]) \oplus (a\otimes [b_0])$

\end{description}
\end{boxedtext}

\smallskip\noindent
The  {\bf return} command has been omitted from the description
as it is assumed that the algorithm returns the last evaluation.
This is true of the {\sl Mathematica\/} language, so 
the {\tt Return[]} statements will be omitted from the programs as well.
The {\sl Mathematica\/} implementation is 

\begin{boxedtext}
\noindent {\small
\begin{verbatim}
a_~plus~b_:= 
If[Length[a] Length[b] == 0, 
   a~join~b,
   Block[{sum = Quotient[#, 10]~join~Mod[#,10]& [last[a] + last[b]]}, 
         ((drop[a]~plus~drop[sum])~plus~drop[b])~join~last[sum]
   ]    ] 
\end{verbatim}
}
\end{boxedtext}

\noindent
where \verb+Quotient[a, b]+ means $\lfloor a/b\rfloor$. 

\medskip\noindent
\begin{boxedtext}
{\small
\begin{verbatim}
a_~times~b_:= 
If[Length[b]==1, 
   If[Length[a] == 1, 
      Quotient[#, 10]~join~Mod[#,10]& [last[a] last[b]], 
      b~times~a], 
   ((a~times~drop[b])~join~{0})~plus~(a~times~last[b])
  ]     
\end{verbatim}
}
\end{boxedtext}

\smallskip\noindent
In this case, one can argue that a better choice of computer
language would be Lisp or Scheme since the programs relied
on list operations and recursions, the basic paradigms of these languages.


\subsection*{2. Using a lower level language}

I recently asked an undergraduate class to solve the problem of the
previous section, and then to implement their algorithm.  It turns out
that all of them managed to write correct programs in C or Pascal, but
that none of them wrote correct high level descriptions. 

The students had a harder time writing the high level description
because they wrote low level programs. The following C implementation
does in fact follow the first two algorithms of Section~1 closely, but
this is hidden by code dealing with input, output, and memory
allocation.  Of course, this program will run at least $1,000$ times
faster than the {\sl Mathematica\/} program, but it's clear that speed
is not the point of this exercise. The data structure is an array {\tt
a[]} where {\tt a[0]} denotes the length of {\tt a[]}.  The main
control structure is

\begin{boxedtext}
\noindent
{\small
\begin{verbatim}
#include <stdio.h>

int *plus(int *num_a, int *num_b);
int *times(int *tnum_a, int *tnum_b); 

main()
{int x[1001], y[1001], *z, s;                           /* up to 1000 digits */
 for(x[0] = 1; (x[x[0]]=getchar()-'0') >= 0; x[0]++);        /* first number */
 for(y[0] = 1; (y[y[0]]=getchar()-'0') >= 0; y[0]++);       /* second number */
 z = ((x[x[0]] + '0') == '+')?plus(x, y):times(x, y);     /* add or multiply */
 for(s = 1; s < z[0]; s++) putchar(z[s]+'0');                /* print answer */
}
\end{verbatim}
}
\end{boxedtext}

\noindent
The addition program first finds the longest input string and allocates
memory accordingly

\begin{boxedtext}
\noindent
{\small
\begin{verbatim}
int *plus(int *p_a, int *p_b) 
{int *a, *b, *c, size_a, size_b, k;
 size_a = (a = (p_a[0] > p_b[0])?p_b:p_a)[0]; 
 size_b = (b = (p_a[0] > p_b[0])?p_a:p_b)[0];
 (c = (int *) calloc(size_b+1, sizeof(int)))[0] = size_b + 1;
 k = 0;
 while(size_b > 1) c[size_b] = (k=(--size_a?a[size_a]:0)+b[--size_b]+ k/10)%10;
 c[1] = (k/10)?1:--(c++[0]);                               /* last digit */  
 return(c); 
}
\end{verbatim}
}
\end{boxedtext}

\noindent
The multiplication program uses the fact that the memory allocated by
{\tt calloc} is initialized to zero. Right shifts simply consist of
incrementing the array length.

\begin{boxedtext}
\noindent
{\small
\begin{verbatim}
int *times(int *t_a, int *t_b) 
{int *c, *temp, j, size_a, size_b, k; 
 (c = (int *) calloc(1, sizeof(int)))[0] = 2;                   /* c = 0 */
 temp = (int *) calloc(2001, sizeof(int));  
 for(size_a = t_a[0]-1; size_a; size_a--){
     temp[0] = t_a[0] - size_a + 1;
     for(size_b = t_b[0]-1; size_b; size_b--){
         temp[1] = (k = t_a[size_a] * t_b[size_b])/10;
         temp[2] = k%10;        
         temp[0]++;                                           /* left shift */ 
         c = plus(c, temp);
     }
    }
 if(!c[1]) c[1] = --(c++[0]);                  /* remove leading zero */
 return(c);
}
\end{verbatim}
}
\end{boxedtext}

\subsection*{3. A non trivial example}

Writing an algorithm to do addition and multiplication does not
require the programming language to have any mathematical features and,
as noted above, using Lisp or Scheme would be a good way to go.  I
will now give a more complicated example where such things as
built--in matrix multiplication will be required, so that using a
computer algebra system like Macsyma, Maple or {\sl Mathematica\/}  
is advantageous.

Recall that a 
continued fraction is a generalization of compound fractions like
$\frac{14}{11}=1 \frac{3}{11}$
$$
\eqalign{
\frac{14}{11} &= 1\frac{3}{11} = 1 + \frac{1}{11/3}
 = 1 + \frac{1}{3\frac{2}{3}}
=  1 + \frac{1}{3 + \displaystyle \frac{1}{3/2}}
=
1 + \frac{1}{3 + \displaystyle \frac{1}{1\frac{1}{2}}}
\cr\noalign{\vskip 5pt}
&= 1 + \frac{1}{3 + \displaystyle\frac{1}{1 + \displaystyle 
   \frac{1}{2}}}\,.
\cr
}
$$
In general, every rational number $p/q$ can be written in the 
form
$$
\frac{p}{q} = a_0 + \frac{1}{a_1 + 
     \displaystyle\frac{1}
       {a_2  + {\mbox{$\;$} \atop {\displaystyle \ddots \qquad \atop 
        \displaystyle \qquad  +\frac{1}{a_r}}}
}}
$$
where the  $a_i$, $i>0$, are positive integers and for ease of notation
this is usually written as
$$
\frac{p}{q} = [a_0, a_1,\ldots, a_r ]\,.
$$ 

\bigskip
Consider the problem of adding and multiplying the sequence of digits
of two continued fraction expansions.  
Even multiplying a continued fraction
by 2 is already a time consuming task 
based on ``Hurwitz rules'' \cite[Exercise~4.5.3.14]{Knuth}, so 
the general problem
was declared to be
hopeless by Khinchin, the renowned expert in the field
\cite{Khinchin2}, who wrote:
``There is, however, another and yet more significant
practical demand that the aparatus of continued fractions does not
satisfy at all. Knowing the representations of several numbers, we
would like to be able, with relative ease, to find the representations
of the simpler functions of these numbers (especially, their sum and
product). In brief, for an apparatus to be suitable from a practical
standpoint, it must admit sufficiently simple rules for arithmetical
operations; otherwise, it cannot serve as a tool for calculation.  We
know how convenient systematic fractions are in this respect.  On the
other hand, for continued fractions there are no practically
applicable rules for arithmetical operations; even the problem of
finding the continued fraction for a sum from the continued fraction
representing the addends is exceedingly complicated, and unworkable in
computational practice.''

The problem was  solved in a little known paper of Marshall Hall
\cite{Hall}, but an actual description of the addition and multiplication
algorithm was not given until the work of R.W.~Gosper  \cite{BGS}
\cite{Gosper} \cite[Exercise~4.5.3.15]{Knuth} \cite[p.~78]{Levy}
based on three ideas. The first idea is that a computation like 
$3x$, where $x$ is given as a continued fraction, will rapidly lead
to more complicated forms. For example, computing $3\cdot \frac{14}{11}$
$$
3 \, [1,3,1,2]
=
3\, \left(1 + \frac{1}{[3, 1, 2]}\right)
=
\frac{3 \, [3, 1, 2] + 1}{[3, 1, 2]}
$$ 
and so forth. Gosper's idea was not to try to simplify these forms,
but to analyze the worst that
can happen. It's not hard to see that you always get 
a linear fractional form
$$
\frac{ax+b}{cx+d}\,,
\leqno(*)
$$
where $a,b,c,d$ are 
integers, and $x$ is given as a continued fraction.

Gosper's second idea was to consider the $x$ as a {\sl formal\/}
symbol which could input continued fraction digits into $(*)$. 
In other words, think of $x$ not as representing a rational number,
but as a symbolic quantity that transforms as
$$
x\mapsto q +\frac{1}{x} \,,
\leqno(**)
$$
where $q$ represents the next continued fraction digit. It remains
to see how $(*)$ transforms under $(**)$. A simple computation 
shows that
$$
\frac{ax+b}{cx+d} \mapsto \frac{aqx + b x + a}{cq x + dx + c}\,.
$$
So if the form $(*)$ is represented by a matrix 
${a\; b\choose c\; d}$, then the transformation law corresponding
to inputting a digit is
$$
\pmatrix{a&b\cr c&d }\mapsto 
\pmatrix{aq+b& a \cr cq + d& c} 
=\pmatrix{a& b\cr c & d}\, \pmatrix{q& 1\cr 1& 0}\,.
$$

\medskip
The third idea is to realize that you can output continued fraction
coefficients without having complete knowledge of $x$. Gosper's 
original example was 
$$
\frac{70 x + 29}{12 x + 5}
$$
If  $x>0$, then 
$$
\frac{29}{5} \le \frac{70 x + 29}{12 x + 5} < \frac{70}{12}\,.
$$
This means that 
$$
\frac{70 x + 29}{12 x + 5} 
=
5 + \frac{10 x + 4}{12x + 5}\,.
$$
Similarly, 
$$
\frac{12}{10} \le \frac{12 x + 5}{10 x + 4} < \frac{5}{4}
$$
so 
$$
\frac{12 x + 5}{10 x + 4} =
1 + \frac{2 x + 1}{10 x+ 4}\,,
$$
and
$$
\frac{10x + 4}{2 x + 1} = 4 + \frac{2x}{2x+1}\,.
$$
This means that for {\sl any\/} $x>0$ you get a continued fraction 
expansion
$$
\frac{70 x + 29}{12 x + 5}
=
5 + \frac{1}{\displaystyle 1 + 
\frac{1}{\displaystyle 4 + \frac{1}{\displaystyle (2x+1)/x}}}
$$
In other words, you are able to output 3 continued fraction coefficients
of the form $(10x+25)/(2x+1)$ without knowing too much about $x$. 
(Note that you cannot output a fourth coefficient since 
$(2x+1)/x \mapsto 2 + 1/x$ does not give the correct coefficient
unless $x>1$.) 

In the case when $x$ represents a sequence of continued fraction 
coefficients, the situation is even better because the coefficients
are greater or equal to one. Now, since one is assuming that there
is a continuing sequence of continued fraction coefficients, the 
further assumption that $x> 1$ can be made, so 
being able to output a continued fraction
corresponds to checking that for some integer $n$ 
$$
n\le \frac{a}{c} \le \frac{a+b}{c+d}< n+1\,,
$$
in other words, that
$$
\left\lfloor \frac{a}{c} \right\rfloor 
=
\left\lfloor \frac{a+b}{c+d}\right\rfloor\,,
$$
and when this happens the output will be the common value
$q = \lfloor a/c\rfloor$. The point is that is always happens, given
a sufficient number of digits of $x$. 

It remains to see what happens to the form after this has been output.
This is 
$$ 
\frac{1}{\displaystyle \frac{ax+b}{cx +d} - q}
=
\frac{c x + d}{a x - cq x + b - dq}\,,
$$
and the corresponding matrix transformation is
$$
\pmatrix{a&b\cr c& d}\mapsto \pmatrix{c & d\cr a-cq& b - dq}
= \pmatrix{0& 1\cr 1 & -q}\,  \pmatrix{a& b \cr c & d} \,.
$$ 

\medskip\noindent


All these steps can be combined into an algorithm but first an
algorithm to compute the continued fraction of a rational number must
be written. This essentially follows the
$\frac{14}{11}$ example (ordinary concatenation $[0]\star
[x_0,\ldots,x_r] = [0,x_0,\ldots,x_r]$ will be used from now on).

\medskip\noindent
\begin{boxedtext}
\noindent
{\bf Algorithm to compute the continued fraction expansion of a 
rational number:} Given a rational number $q$, return a sequence
$f(q)$:

\smallskip\noindent
{\bf if} $q=1/0$ {\bf then} $[\;]$
{\bf else} $[\lfloor q\rfloor]\star f(1/(q-\lfloor q\rfloor))$
\end{boxedtext}

 
\bigskip\noindent
\begin{boxedtext}
\noindent
{\bf Algorithm to compute the continued fraction expansion of
$(ax+b)/(cx+d)$:} Given a matrix ${a\; b\choose c \; d}$ and the
continued fraction expansion $x = [x_0, \ldots, x_r]$, 
return the sequence $f({a\; b\choose c\; d}, x)$:

\begin{description}
\item[{\bf if}] $r=0$ {\bf then}  $f((a\, x_0+b)/(c \, x_0 + d))$.

\item[{\bf else}] 

\begin{description}

\item[{\bf if}] 
$a,b,c,d>0$ and 
$\left \lfloor \frac{a}{c} \right \rfloor =
\left \lfloor \frac{a+b}{c+d} \right \rfloor 
$ {\bf then}
$[\lfloor a/c\rfloor]\star  
 f\left({0\;\;\;\;\;\;\;\;\; 1 \choose 1\;\; -\lfloor a/c\rfloor }
  {a\; b\choose c\; d}, x\right)$ (Output a digit.)

\item[{\bf else}] 
$f\left({a\; b\choose c\; d} \, 
 {x_0 \; 1\choose 1\; \; 0}, [x_1,\ldots,x_r]\right)$ \quad (Input a digit.)
\end{description} 

\end{description}
\end{boxedtext}

\smallskip\noindent
In {\sl Mathematica\/} these algorithms are 

\medskip\noindent
\begin{boxedtext}
\noindent
{\small
\begin{verbatim}
ContinuedFraction[q_]:= 
If[q == 1/0,{},{Floor[q]}~Join~ContinuedFraction[1/(q-Floor[q])]

ContinuedFraction[matrix_, x_]:= 
If[Length[x] == 1, 
   ContinuedFraction[Divide @@ (matrix . {First[x], 1})], 
   Block[{q = Positive[Min[matrix]] && Quotient @@ (matrix . {{1, 1}, {0, 1}})},
         If[q && Min[q] == Max[q],
            {#}~Join~ContinuedFraction[{{0, 1}, {1, -#}} . matrix, x]& [First[q]],  
            ContinuedFraction[matrix . {{First[x], 1}, {1, 0}}, Rest[x]]  
   ]    ]  ]
\end{verbatim}
}
\end{boxedtext}

\smallskip\noindent
Since division by zero has been allowed the line

{\small
\begin{verbatim}
Off[Power::infy]
\end{verbatim}
}

\noindent
is useful in order to avoid annoying warning messages. 

Note the similarity in the control structure of the program and the
pseudo code. It should be clear that it was essentially a
transcription of the computer program.

\bigskip
Finally, the algorithms for addition and multiplication can be 
described. 
Just as before computing  $x+y$ or 
$xy$ leads to more complicated expressions. 
The worst expression you get is a ``bilinear fractional form'' 
$$
\frac{a xy + b x + c y + d}{e xy + f x + g y + h}\,.
\leqno(***)
$$
Treating $x$ and $y$ as formal variables, this can be represented
by an ordered pair of matrices (or tensor)
$$
\left(\pmatrix{a& b\cr c & d}, \pmatrix{e& f\cr g & h }\right)\,.
$$
Letting $y \mapsto q + 1/y$ in $(***)$ gives a transformation for the
tensor
$$
\left(\pmatrix{a& b\cr c & d}, \pmatrix{e& f\cr g & h }\right)
\mapsto 
\left(\pmatrix{a& b\cr c & d}, \pmatrix{e& f\cr g & h }\right)\, 
\pmatrix{q & 1\cr 1 & 0}\,,
$$
where ordinary matrix multiplication is used on each component. 
Similarly, letting $x\mapsto q + 1/x$ in $(***)$ corresponds to 
$$
\left(\pmatrix{a& b\cr c & d}, \pmatrix{e& f\cr g & h }\right)
\mapsto 
\pmatrix{q & 1\cr 1 & 0}\,
\left(\pmatrix{a& b\cr c & d}, \pmatrix{e& f\cr g & h }\right)\,. 
$$
Just as before, one uses an Euclidean algorithm to output coefficients,
but this time there won't be a guarantee that intermediate values 
will be greater than one. A tricky part of the algorithm is
to decide whether to choose $x$ or $y$ to input the next digit. 
A direct solution to this is to alternate between them and
this requires a componentwise transpose
$$
\left(\pmatrix{a& b\cr c & d}, \pmatrix{e& f\cr g & h }\right)^T
=
\left(\pmatrix{a& c\cr b & d}, \pmatrix{e& g\cr f & h }\right)\,,
$$
corresponding to switching $x$ and $y$ in $(***)$. 


\medskip\noindent
\begin{boxedtext}
\noindent
{\bf Algorithm to compute the continued fraction of 
 $(a xy + b x + c y + d)/(e xy + f x + g y + h)$:}
Given as input a tensor 
$\left({a\; b\choose c \; d}, {e\; f\choose g \; h }\right)$
and continued fraction expansions 
$x = [x_0,\ldots,x_r]$, $y=[y_0,\ldots,y_s]$, return a sequence
$f\left(\left({a\; b\choose c \; d}, {e\; f\choose g \; h }\right),x,y\right)$:

\begin{description}

\item[{\bf if}]
$s = 0$ {\bf then} 
$f\left(\left({a\; b\choose c \; d}, {e\; f\choose g \; h }\right)\, 
   {y_0 \choose 1}, x\right)$. 

\item[{\bf else}]

\begin{description}

\item[{\bf if}]
 $a,b,c,d,e,f,g,h > 0$ and 
$\lfloor a/e\rfloor= \lfloor b/f\rfloor= 
 \lfloor c/g\rfloor =\lfloor d/h\rfloor$ {\bf then} 
$$
\textstyle 
[\lfloor a/e\rfloor]\star 
f\left({0\;\;\;\;\;\;\;\;\; \; 1\choose 1 \;\; -\lfloor a/e\rfloor } \, 
 \left({a \; b\choose c \; d}, {e \; f\choose g \; h}\right), 
        x, y\right) \qquad\mbox{(Output a digit.)}
$$ 

\item[{\bf else}]
$f\left(
\left\{\left({a \; b\choose c \; d}, {e \; f\choose g \; h}\right)
\, {y_0\;  1\choose 1\;\;\, 0}\right\}^T,
      [y_1, \ldots,y_s], x\right)$ \quad (Input a digit and switch $x,y$.)
\end{description}
\end{description}
\end{boxedtext}

\medskip\noindent
This algorithm allows one to add and multiply continued fractions
according to the rules
$$
x\oplus y  = \textstyle 
f\left(\left({0 \; 1\choose 1\; 0}, {0 \; 0 \choose 0 \; 1}\right), x,y
   \right)\,,\qquad
x\otimes y  = \textstyle 
f\left(\left({1 \; 0\choose 0\; 0}, {0 \; 0 \choose 0 \; 1}\right), x,y
   \right)\,.
$$

\medskip\noindent
The program is simplified by the fact that ordinary {\sl
Mathematica\/} matrix multiplication for generalized matrices behaves
in the same way as the componentwise matrix multiplication defined
above.

\medskip\noindent
\begin{boxedtext}
\noindent
{\small
\begin{verbatim}
ContinuedFraction[tensor_, x_, y_]:= 
If[Length[y] == 1, 
   ContinuedFraction[tensor.{First[y], 1}, x],
   Block[{q = Positive[Min[tensor]] && Quotient @@ tensor},
          If[q && Min[q] == Max[q], 
             {#}~Join~ContinuedFraction[{{0, 1},{1, -#}}.tensor, x, y]& [q[[1,1]]],
              ContinuedFraction[Transpose /@ (tensor.{{First[y], 1},{1,0}}), 
                                Rest[y], x]]
   ]       ]

ContinuedFractionPlus[x_, y_]:= 
ContinuedFraction[{{{0,1},{1,0}},{{0,0},{0,1}}}, x, y]

ContinuedFractionTimes[x_, y_]:= 
ContinuedFraction[{{{1,0},{0,0}},{{0,0},{0,1}}}, x, y]
\end{verbatim}
}
\end{boxedtext}

\medskip\noindent
{\bf Remark 1:} Subtraction and division can be also be computed 
in this way by using the formulas
$$
x\ominus y  = \textstyle 
f\left(\left({0\;\;\; 1\choose -1\; 0}, {0 \; 0 \choose 0 \; 1}\right), x,y
   \right)\,,\qquad
x\oslash y  = \textstyle 
f\left(\left({0 \; 1\choose 0\; 0}, {0 \; 0 \choose 1 \; 0}\right), x,y
   \right)\,.
$$

\medskip\noindent
{\bf Remark 2:} The real advantage of Gosper's method is that it allows
you to add and multiply numbers with known continued fraction
expansions, for example, $e = [2,1,2,1,1,4,1,1,6,\ldots]$ and 
$\phi = \frac{\sqrt{5}+1}{2} = [1,1,1,\ldots]$. The above algorithms
are easily modified to compute quantities like
$e+\phi$ where the digit input would be the functions
$$
e(n) = \cases{2 & if $n=0$ \cr 2(n+1)/3 & if $n\equiv 2 \; ({\rm mod}\; 3)$\cr
              1 & otherwise,}\, 
\qquad  {\rm and }\qquad 
\phi(n) = 1\,. 
$$
 

\medskip\noindent
{\bf Remark 3:} There are problems with continued fraction 
 arithmetic. For example, 
the computation $[1,2,2,2,\ldots]\otimes [1,2,2,2,\ldots]$ 
corresponding to  $\sqrt{2}\times \sqrt{2}$, will not return a single
correct continued fraction coefficient or return $[2,a]$, where $a$
is a large integer, depending on whether one uses an odd or even 
length input. This is slightly different from the decimal case.

\begin{thebibliography}{xxxxx}

\bibitem{BGS}
M. Beeler, R.W. Gosper, and R. Schroeppel, {\sl  HAKMEM,} 
 A.I. Lab Memo \# 239, M.I.T. 1972. 

\bibitem{Gonnet}
G.H. Gonnet and R. Baeza--Yates, {\sl Handbook of Algorithms and 
Data Structures,} Second Edition, Addison--Wesley 1991. 

\bibitem{Gosper}
R.W. Gosper, {\sl Continued Fraction Arithmetic,} preprint 1976.

\bibitem{Hall}
M. Hall, Jr., {\sl On the sum and product of continued fractions,}
Annals of Math. {\bf 48} (1947), 966--993.

\bibitem{Khinchin2}
A. Ya. Khinchin, {\sl Continued Fractions,} University of Chicago Press,
Chicago 1964.

\bibitem{Knuth}
D.E. Knuth, {\sl The Art of Computer Programming, Vol. 2, Seminumerical
Algorithms,} Addison--Wesley 1981.

\bibitem{Levy}
S. Levy, {\sl Hackers: heroes of the computer revolution,} Doubleday, 1984.

\bibitem{Sedgewick}
R. Sedgewick, {\sl Algorithms,} Second Edition, Addison--Wesley 1988.

\bibitem{SedgewickC}
R. Sedgewick, {\sl Algorithms in C}, Addison--Wesley 1990.

\end{thebibliography}

\end{document}
